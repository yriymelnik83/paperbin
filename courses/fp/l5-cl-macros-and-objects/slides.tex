\documentclass{beamer}

\usepackage{amsmath}
\usepackage[style=alphabetic,url=true]{biblatex}
\usepackage{environ}
\usepackage{geometry}
\usepackage{graphicx}
\usepackage{amsmath}
\usepackage{amsfonts}
\usepackage{amssymb}
\usepackage{calrsfs}
\usepackage{listings}
\usepackage{tikz}
\usepackage[T2A]{fontenc}
\usepackage[utf8]{inputenc}
\usepackage[cache=false]{minted}

\graphicspath{ {./graphics/} }
\usetikzlibrary{shapes.arrows,chains}
\usecolortheme{beaver}
\setbeamertemplate{itemize item}[circle]
\setbeamertemplate{itemize subitem}{--}
\addtobeamertemplate{navigation symbols}{}{
  \usebeamerfont{footline}
  \usebeamercolor[fg]{footline}
  \hspace{1em}
  \insertframenumber/\inserttotalframenumber
}
\setminted[Lisp]{
  fontsize=\scriptsize
}
\BeforeBeginEnvironment{minted}{\medskip}
\AfterEndEnvironment{minted}{\medskip}



\title{
  Common Lisp and Introduction to Functional Programming \\
  Lecture 5: CLOS, MOP and Macros
}
\author{Yuri Zhykin}
\date{Mar 11, 2021}

\begin{document}

\frame{\titlepage}

\begin{frame}[fragile]
  \frametitle{Common Lisp Object System}
  \begin{itemize}
  \item \textbf{CLOS, Common Lisp Object System} originally was a ``plug-in''
    object system for the Common Lisp language, implemented as a library.
  \item Final version of CLOS is included in the ANSI Common Lisp standard and
    well integrated into the language.
  \item CLOS is inspired by the earlier Lisp object systems \textbf{MIT Flavors}
    and \textbf{CommonLoops}.
  \item CLOS is considered one of the most powerful and flexible object systems
    in mainstream programming languages.
  \end{itemize}
\end{frame}

\begin{frame}[fragile]
  \frametitle{CLOS entities}
  \begin{itemize}
  \item \textbf{Classes} and \textbf{Objects}.
  \item \textbf{Generic Functions}.
  \item \textbf{Methods}.
  \item \textbf{Specializers}.
  \item \textbf{Method Combinations}.
  \end{itemize}
\end{frame}

\begin{frame}[fragile]
  \frametitle{Classes and Objects 1/2}
  \begin{itemize}
  \item Classes can be declared with the \mintinline{Lisp}{defclass} form:
\begin{minted}{Lisp}
  (defclass <class-name> (<superclass-1> <superclass-2> ...)
    ((<slot-1> 
      :accessor <class-slot-1>
      :initarg  :slot-1
      :initform <slot-1-initform>)
     ...)
    <options>)
\end{minted}
  \item Instances of classes (\textbf{objects}) can be created with:
\begin{minted}{Lisp}
  CL-USER> (make-instance <class-name-symbol>
                          :slot-1 <slot-1-value> ...)
  #<<class-name> :slot-1 <slot-1-value> ...>
\end{minted}    
  \end{itemize}
\end{frame}

\begin{frame}[fragile]
  \frametitle{Classes and Objects 2/3}
  \begin{itemize}
  \item Authenticator class hierarchy example:
\begin{minted}{Lisp}
  (defclass auth ()
    ((user :accessor auth-user :initarg :user)))

  (defclass secret-auth (auth)
    ((secret :accessor auth-secret :initarg :secret)))

  (defclass password-auth (secret-auth) ())

  (defclass token-auth (secret-auth) ())
\end{minted}
  \end{itemize}
\end{frame}

\begin{frame}[fragile]
  \frametitle{Classes and Objects 3/3}
  \begin{itemize}
  \item CLOS classes support multiple inheritance.
  \item Common Lisp provides another type of record entities - \textbf{structs}:
\begin{minted}{Lisp}
  CL-USER (defstruct point x y)
  point
  CL-USER> (make-point :x 1 :y 2)
  #S(point :x 1 :y 2)
  CL-USER> (point-x *)
  1
\end{minted}
  \item Structs are also classes, but they are easier to use in some cases and
    may have different internal representation.
  \end{itemize}
\end{frame}

\begin{frame}[fragile]
  \frametitle{Generic Functions}
  \begin{itemize}
  \item \textbf{Generic function} is an object that consists of a set of all
    \textbf{methods} that have the same symbolic name.
  \item \textit{Generic function declarations} are used to fix the
    \textit{contracts} (\textit{signatures}) of its methods and to store
    documentation.
  \item For example:
\begin{minted}{Lisp}
  (defgeneric authenticate (auth)
    (:documentation "Authenticate the user with the given AUTH."))
\end{minted}
  \end{itemize}
\end{frame}

\begin{frame}[fragile]
  \frametitle{Methods}
  \begin{itemize}
  \item \textbf{Methods} are functions that implement a generic function for a
    particular set of arguments.
  \item Methods are added to the method set of a generic function with the same
    name.
  \item If generic function with a given name does not exist, it is created
    automatically.
  \item Example:
\begin{minted}{Lisp}
  (defmethod authenticate ((auth auth))
    nil)

  (defmethod print-object ((auth secret-auth) stream)
    (format stream "#<SECRET-AUTH(~a)>" (auth-user auth)))

  (defmethod authenticate ((auth password-auth))
    (when (equal (sha512 (auth-secret auth)) <stored-pw-hash>)
      ...))

\end{minted}    
  \end{itemize}
\end{frame}

\begin{frame}[fragile]
  \frametitle{Specializers}
  \begin{itemize}
  \item \textit{Applicability} of methods is defined by \textbf{specializers} -
    entities that describe a property of an argument that is sufficient to match
    a given method.
  \item Current Common Lisp standard defines two types of specializers
    \begin{itemize}
    \item \textbf{type} - class/inheritance tree of the object is used,
    \item \textbf{eql} - object is tested for equality with a given specializer.
    \end{itemize}
  \item Example of \mintinline{Lisp}{eql}-specializer:
\begin{minted}{Lisp}
  (defmethod authenticate ((auth (eql :dummy-auth)))
    t)
\end{minted}
  \end{itemize}
\end{frame}

\begin{frame}[fragile]
  \frametitle{Method combinations 1/2}
  \begin{itemize}
  \item \textbf{Method combinations} allow to define how \textit{all} applicable
    methods will be combined into an \textbf{effective method}.
  \item \textbf{Standard method combination} defines the notions of
    \textbf{primary method} and \textbf{auxiliary methods}.
  \item Auxiliary methods can be of the following types:
    \begin{itemize}
    \item \textbf{before} - called before primary method; result discarded,
    \item \textbf{after} - called after primary method; result discarded,
    \item \textbf{around} - called around before/primary/after method.
    \end{itemize}
  \item \mintinline{Lisp}{:around}-methods must call
    \mintinline{Lisp}{call-next-method} function or otherwise method call
    sequence will be terminated.
  \end{itemize}
\end{frame}

\begin{frame}[fragile]
  \frametitle{Method combinations 2/2}
  \begin{itemize}
  \item Standard method combination example:
\begin{minted}{Lisp}
  (defmethod authenticate :before ((auth token-auth))
    (load-token-storage))

  (defmethod authenticate :after ((auth token-auth))
    (unload-token-storage))

  ;; Alternatively:
  (defmethod authenticate :around ((auth token-auth))
    (load-token-storage)
    (prog1
      (call-next-method)
      (unload-token-storage)))
\end{minted}
  \end{itemize}
\end{frame}

\begin{frame}[fragile]
  \frametitle{Metaobject Protocol}
  \begin{itemize}
  \item \textbf{Metaobject} is an object that manipulates, creates, describes,
    or implements objects (including itself).
  \item \textbf{Metaobject protocol} is an object-based model that consists of
    \textit{metaobjects} and allows to manipulate the structure and behaviour of
    an object system:
    \begin{itemize}
    \item create or delete new classes,
    \item create new generic functions and methods,
    \item define class slot access,
    \item manipulate class inheritance relations,
    \item manipulate the \textbf{effective method} computation.
    \end{itemize}
  \item Metaobject protocol can be used to create an additional object system
    that better fits the domain of the software.
  \end{itemize}
\end{frame}

\begin{frame}[fragile]
  \frametitle{Class Precedence Lists}
  \begin{itemize}
  \item Some of the MOP utilities are available in standard Common Lisp but most
    of MOP is usually available as an additional package:
\begin{minted}{Lisp}
  CL-USER> (find-class 'password-auth)
  #<standard-class password-auth>
\end{minted}
  \item One of the most important MOP concepts is \textbf{class precedence
      list}:
\begin{minted}{Lisp}
  CL-USER> (mop:class-precedence-list *)
  (#<standard-class password-auth> #<standard-class secret-auth>
   #<standard-class auth> #<standard-class standard-object>
   #<built-in-class t>)
\end{minted}
  \end{itemize}
\end{frame}

\begin{frame}[fragile]
  \frametitle{Effective Methods}
  \begin{itemize}
  \item \textit{Class precedence lists} are used when computing
    \textbf{effective methods}.
  \item \textbf{Effective method} is a function that is actually called when a
    generic function is called.    
  \item \textit{Effective method} is a \textit{closure} that combines a subset
    of \textit{applicable methods} based on \textit{class precedence} and
    \textit{method combinations}:
\begin{minted}{Lisp}
  (lambda (&rest args)
    (if <around-methods-exists-p>
        <call-around-methods> ;; calls before/primary/after-methods
        (progn
          <call-before-method>
          <call-primary-methods>
          <call-after-method>))
\end{minted}    
  \end{itemize}
\end{frame}

\begin{frame}[fragile]
  \frametitle{Macros}
  \begin{itemize}
  \item \textbf{Macros} are special forms that expand into actual Lisp forms
    during compilation.
  \item Powerful macro system allows to extend the syntax of the language with
    the new special forms and control structures.
  \item More primitive languages (e.g. \textit{C}) use macros that generate
    text via simple textual substitutions.
  \item Lisp macros generate Lisp forms represented as data structures (lists),
    and any Lisp code can be executed when computing the macro expansion.
  \item \textbf{Common Lisp macros are Common Lisp functions that are called
      during compilations and return Common Lisp code represented as trees that
      will be further compiled.}
  \end{itemize}
\end{frame}

\begin{frame}[fragile]
  \frametitle{Read/Compile/Load/Evaluate Phases}
  \begin{itemize}
  \item \textbf{Read} phase reads \textit{textual} representation of Lisp and
    returns a tree representation for each \textit{top-level} form.
  \item \textbf{Compile} phase recursively expands macros until there is nothing
    to expand and compiles the result Lisp code.
  \item \textbf{Load} phase loads the compiled Lisp code (usually in the form of
    \mintinline{Lisp}{*.fasl} files into the running Lisp executing all
    top-level forms.
  \item \textbf{Execute} phase is started when a Lisp form is invoked within a
    Lisp system (e.g. from REPL).
  \end{itemize}
\end{frame}

\begin{frame}[fragile]
  \frametitle{Lisp Code that Returns Lisp Code 1/2}
  \begin{itemize}
  \item Simplest Lisp macros can simply construct and return lists that
    represent the resulting expansion:
\begin{minted}{Lisp}
  (defmacro %when (condition &body true-body)
    (list 'if condition (cons 'progn true-body) nil))

  CL-USER> (macroexpand '(%when t (print "true") t))
  (if t (progn (print "true") t) nil)
  t
\end{minted}
  \end{itemize}
\end{frame}

\begin{frame}[fragile]
  \frametitle{Lisp Code that Returns Lisp Code 2/2}
  \begin{itemize}
  \item More concise way to construct Lisp forms is to use the
    \textbf{quasi-quote}:
\begin{minted}{Lisp}
  (defmacro %when (condition &body true-body)
    `(if ,condition 
         (progn ,@true-body)
         nil))

  CL-USER> (macroexpand '(%when t (print "true") t))
  (if t (progn (print "true") t) nil)
  t
\end{minted}
  \end{itemize}
\end{frame}

\begin{frame}[fragile]
  \frametitle{Why Macros?}
  \begin{itemize}
  \item Can the \mintinline{Lisp}{if}-form be implemented as a function?
\begin{minted}{Lisp}
  (if (some-condition-p)
      (print "true")
      (print "false")
\end{minted}
  \end{itemize}
\end{frame}

\begin{frame}
  \frametitle{Useful Resources}
  \begin{itemize}
  \item \textbf{The Art of the Metaobject Protocol} by Daniel G. Bobrow and
    Gregor Kiczales - one of the best OOP books out there
  \item \textbf{Let Over Lambda} by Doug Hoyte - great book on macro programming
    in Common Lisp
  \end{itemize}
\end{frame}

\begin{frame}
  \frametitle{The End}
  \begin{center}
    Thank you!
  \end{center}
\end{frame}

\end{document}

%%% Local Variables:
%%% mode: latex
%%% TeX-master: t
%%% TeX-command-extra-options: "-shell-escape"
%%% End:
