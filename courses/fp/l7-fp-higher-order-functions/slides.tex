\documentclass{beamer}

\usepackage{amsmath}
\usepackage[style=alphabetic,url=true]{biblatex}
\usepackage{environ}
\usepackage{geometry}
\usepackage{graphicx}
\usepackage{amsmath}
\usepackage{amsfonts}
\usepackage{amssymb}
\usepackage{calrsfs}
\usepackage{listings}
\usepackage{tikz}
\usepackage[T2A]{fontenc}
\usepackage[utf8]{inputenc}
\usepackage[cache=false]{minted}

\graphicspath{ {./graphics/} }
\usetikzlibrary{shapes.arrows,chains}
\usecolortheme{beaver}
\setbeamertemplate{itemize item}[circle]
\setbeamertemplate{itemize subitem}{--}
\addtobeamertemplate{navigation symbols}{}{
  \usebeamerfont{footline}
  \usebeamercolor[fg]{footline}
  \hspace{1em}
  \insertframenumber/\inserttotalframenumber
}
\setminted[Lisp]{
  fontsize=\scriptsize
}
\BeforeBeginEnvironment{minted}{\medskip}
\AfterEndEnvironment{minted}{\medskip}



\title{
  Common Lisp and Introduction to Functional Programming \\
  Lecture 7: Higher-order Functions
}
\author{Yuri Zhykin}
\date{Mar 25, 2021}

\begin{document}

\frame{\titlepage}

\begin{frame}[fragile]
  \frametitle{So What Is Functional Programming? 1/3}
  \begin{itemize}
  \item \textbf{Procedural programming} is an \textbf{imperative} programming
    paradigm in which ``procedure'' definitions a sequence of imperative
    statements which update the \textbf{state} of the program.
  \item \textbf{Functional programming} is a \textbf{declarative} programming
    paradigm in which ``function'' definitions are \textbf{trees of expressions}
    that map values to other values.
  \item \textbf{Functional programming} is a programming paradigm where programs
    are constructed by \textbf{applying} and \textbf{composing} functions.
  \end{itemize}
\end{frame}

\begin{frame}[fragile]
  \frametitle{So What Is Functional Programming? 2/3}
  \begin{itemize}
  \item Functional programming approach builds software by composing
    \textbf{pure functions}, avoiding \textbf{shared state}, \textbf{mutable
      data}, and \textbf{side-effects}.
  \item Program state is not modified explicitly but rather ``flows'' through
    function calls.
  \item Functional program can be a single expression consisting of a large
    number of function invocations.
  \item State is passed as arguments to functions-components of the program
    expression and returned as results, which become arguments to other
    functions-components.
  \end{itemize}
\end{frame}

\begin{frame}[fragile]
  \frametitle{So What Is Functional Programming? 3/3}
  \begin{itemize}
  \item Example of a functional ``data pipeline'':
\begin{minted}{Lisp}
  (defun collect-data (sources)
    "Returns raw data."
    ...)

  (defun process-data (data)
    "Processes raw data and returns results."
    ...)

  (defun analyse-results (results)
    "Analyses results and returns analysis data."
    ...)

  (defun show-analysis (analysis)
    "Prints analysis data to standard output."
    ...)

  (defun main ()
    (let ((sources (list <source1> <source2> ...)))
      (show-analysis 
       (analyse-results (process-data (collect-data sources))))))
\end{minted}
  \end{itemize}
\end{frame}


\begin{frame}[fragile]
  \frametitle{First-class Functions}
  \begin{itemize}
  \item A programming language is said to have \textbf{first-class functions} if
    it treats functions as \textbf{first-class values}.
  \item \textbf{First-class values} or language's \textbf{first-class citizens}
    are values that can be
    \begin{itemize}
    \item passed as arguments to functions,
    \item returned as results from functions,
    \item assigned to variables,
    \item stored in data structures.
    \end{itemize}
  \item \textbf{First-class functions enable passing behaviour as arguments to
      functions.}
  \end{itemize}
\end{frame}

\begin{frame}[fragile]
  \frametitle{Higher-order Functions 1/2}
  \begin{itemize}
  \item \textbf{Higher-order function} (a.k.a., \textbf{Operator} or
    \textbf{Functional} in mathematics) is a function that takes one or more
    functions as arguments or returns a function as its result.
  \item Functions that neither take functions as parameters nor return functions
    as results, are called \textbf{first-order functions}.
  \item Higher-order functions are necessary in order to have first-class
    functions.
  \end{itemize}
\end{frame}

\begin{frame}[fragile]
  \frametitle{Higher-order Functions 2/2}
  \begin{itemize}
  \item \textbf{Differential operator} is a common example of higher-order
    functions: it takes function as an argument and maps it to its derivative,
    which is also a function
    $$\dfrac{d}{dx} f(x) = f'(x)$$
  \item Common Lisp's \mintinline{Lisp}{funcall} and \mintinline{Lisp}{apply}
    are also higher-order functions, since they take functions as their first
    argument:
\begin{minted}{Lisp}
  CL-USER> (funcall (lambda (x) (* x 2)) 5)
  10
  CL-USER> (apply #'+ '(1 2 3 4 5))
  15
\end{minted}
  \end{itemize}
\end{frame}

\begin{frame}[fragile]
  \frametitle{Functional Composition 1/2}
  \begin{itemize}
  \item Main role of functions, similar to procedures in procedural programming,
    is to group computations into units that have a clearly defined common goal
    and dependencies - arguments.
  \item In order to construct programs, one must be able to \textbf{apply} and
    \textbf{compose} functions.
  \item Common Lisp provides operators for applying functions
    (\mintinline{Lisp}{funcall}, \mintinline{Lisp}{apply}), while composing is
    trivially done via nested expressions:
\begin{minted}{Lisp}
  (f1 (f2 a1 a2) (f3 (f4 a3)))
\end{minted}
  \end{itemize}
\end{frame}

\begin{frame}[fragile]
  \frametitle{Functional Composition 2/2}
  \begin{itemize}
  \item \textbf{Function composition operator} is a binary operator that takes
    two functions $f$ and $g$ and returns a function $h$ such that
    $h(x) = f(g(x))$:
\begin{minted}{Lisp}
  (defun compose (&rest fs)
    (if (null fs)
        (lambda (x) x)
        (lambda (x) 
          (funcall (car fs) (funcall (apply #'compose (cdr fs)) x)))))
\end{minted}
  \item Function composition operator enables very elegant style of programming
    called \textbf{tacit style} or \textbf{point-free style}, in which function
    definitions do not identify their arguments:
\begin{minted}{Lisp}
  (defun main ()
    (apply (compose #'show-analysis
                    #'analyse-results
                    #'process-data
                    #'collect-data)
           (list <source1> <source2> ...)))
\end{minted}
  \end{itemize}
\end{frame}

% \begin{frame}[fragile]
%   \frametitle{Currying and Partial Application}
%   \begin{itemize}
%   \item \textbf{Currying} is the technique of converting a function that takes
%     multiple arguments into a sequence of functions that each take a single
%     argument:
%     $$f(x, y, z) \rightarrow \begin{cases}
%       h = &g(x) \\
%       i = &h(y) \\
%       f = &i(z) \\
%     \end{cases}$$
%     $$f(x, y, z) = g(x)(y)(z)$$
%   \item \textbf{Currying} transforms a function of multiple arguments into a set
%     of functions of a single argument, and can be used practically for
%     constructing new functions and theoretically for simplified analysis of
%     functions:
% \begin{minted}{Lisp}
%   (defun curry (f x)
%     (lambda (y) (funcall f x y)))
  
%   (funcall (curry (curry #'+ 1) 2) 3)
% \end{minted}
%   \end{itemize}
% \end{frame}

\begin{frame}[fragile]
  \frametitle{Map, Reduce and Filter}
  \begin{itemize}
  \item \textbf{Recursive operators} \mintinline{Lisp}{Map},
    \mintinline{Lisp}{Filter} and \mintinline{Lisp}{Reduce} are higher-order
    functions that take an recursively defined object and a function, and apply
    that function along the structure of the object.
  \item Common Lisp provides standard functions \mintinline{Lisp}{mapcar},
    \mintinline{Lisp}{remove-if-not} and \mintinline{Lisp}{reduce} that operate
    on lists.
  \item Similar functions that operate on sequences are available in most modern
    programming languages (Java Script, Python, Java, etc).
  \item Just using these three operators may be enough for most sequence
    processing tasks.
  \end{itemize}
\end{frame}

\begin{frame}[fragile]
  \frametitle{Map, Filter and Reduce: Map}
  \begin{itemize}
  \item Function \mintinline{Lisp}{Map} takes a function $f: A \mapsto B$ and a
    list $A$ and \textbf{returns a new list $B$} such that
    $$\forall a_i \in A, b_i \in B: b_i = f(a_i)$$
  \item Example:
\begin{minted}{Lisp}
  CL-USER> (mapcar (lambda (x) (* x x)) '(1 2 3 4 5))
  (1 4 9 16 25)
\end{minted}
  \end{itemize}
\end{frame}

\begin{frame}[fragile]
  \frametitle{Map, Filter and Reduce: Filter}
  \begin{itemize}
  \item Function \mintinline{Lisp}{filter} takes a function
    $f: A \mapsto \{True, False\}$ and a List $A$ and \textbf{returns a new list
      $B$} such that
    $$\forall a_i \in A: a_i \in B \iff f(a_i) = True$$
  \item Example:
\begin{minted}{Lisp}
  CL-USER> (remove-if-not #'evenp '(1 2 3 4 5))
  (2 4)
\end{minted}
  \end{itemize}
\end{frame}

\begin{frame}[fragile]
  \frametitle{Map, Filter and Reduce: Reduce}
  \begin{itemize}
  \item Function \mintinline{Lisp}{reduce} takes a function
    $f: B \times A \mapsto B$, a list $A = {a_1, a_2, ..., a_k}$ and an initial
    value $b_0$ and \textbf{returns a value $b_k$} where
    $$b_i = f(b_{i-1}, a_i), i=1,2,...,k$$
  \item Example:
\begin{minted}{Lisp}
  CL-USER> (reduce #'+ '(1 2 3 4 5) :initial-value 0)
  15
  CL-USER> (+ (+ (+ (+ (+ 0 1) 2) 3) 4) 5)
  15
\end{minted}
  \end{itemize}
\end{frame}

\begin{frame}[fragile]
  \frametitle{Combining Map, Filter and Reduce 1/2}
  \begin{itemize}
  \item Typical data pipeline with \mintinline{Lisp}{Map},
    \mintinline{Lisp}{Filter} and \mintinline{Lisp}{Reduce} looks like this:
\begin{minted}{Lisp}
  ;; (aggregate (transform (select (collect ...))))
  (reduce (lambda (x y) ...)
          (mapcar (lambda (x) ...)
                  (remove-if-not (lambda (x) ...)
                                 (collect ...))))
\end{minted}
  \item This general pattern fits large number of use cases and sequence
    processing problems and can be applied almost identically in most of modern
    programming languages.
  \end{itemize}
\end{frame}

\begin{frame}[fragile]
  \frametitle{Combining Map, Filter and Reduce 2/2}
  \begin{itemize}
  \item Example:
\begin{minted}{Lisp}
  (flet (;; Return a number of exported functions for a package.
         (count-exported-functions (p)
           (let ((fcount 0))
             (do-external-symbols (s p) 
               (when (fboundp s) (incf fcount)))
             fcount))

         ;; Check if a package is an SBCL-supplied one.
         (sbcl-package-p (p)
           (let ((index (search "SB-" (package-name p))))
             (and index (zerop index)))))

    ;; Collect, filter, process and aggregate data.
    (reduce 
     #'+ 
     (mapcar #'count-exported-functions
             (remove-if-not #'sbcl-package-p (list-all-packages)))
     :initial-value 0))
\end{minted}
  \end{itemize}
\end{frame}

\begin{frame}
  \frametitle{The End}
  \begin{center}
    Thank you!
  \end{center}
\end{frame}

\end{document}

%%% Local Variables:
%%% mode: latex
%%% TeX-master: t
%%% TeX-command-extra-options: "-shell-escape"
%%% End:
