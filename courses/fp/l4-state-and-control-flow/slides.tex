\documentclass{beamer}

\usepackage{amsmath}
\usepackage[style=alphabetic,url=true]{biblatex}
\usepackage{environ}
\usepackage{geometry}
\usepackage{graphicx}
\usepackage{amsmath}
\usepackage{amsfonts}
\usepackage{amssymb}
\usepackage{calrsfs}
\usepackage{listings}
\usepackage{tikz}
\usepackage[T2A]{fontenc}
\usepackage[utf8]{inputenc}
\usepackage[cache=false]{minted}

\graphicspath{ {./graphics/} }
\usetikzlibrary{shapes.arrows,chains}
\usecolortheme{beaver}
\setbeamertemplate{itemize item}[circle]
\setbeamertemplate{itemize subitem}{--}
\addtobeamertemplate{navigation symbols}{}{
  \usebeamerfont{footline}
  \usebeamercolor[fg]{footline}
  \hspace{1em}
  \insertframenumber/\inserttotalframenumber
}
\setminted[Lisp]{
  fontsize=\scriptsize
}
\BeforeBeginEnvironment{minted}{\medskip}
\AfterEndEnvironment{minted}{\medskip}



\title{
  Common Lisp and Introduction to Functional Programming \\
  Lecture 4: State and Control Flow
}
\author{Yuri Zhykin}
\date{Feb 17, 2021}

\begin{document}

\frame{\titlepage}

\begin{frame}[fragile]
  \frametitle{Program State}
  \begin{itemize}
  \item \textbf{Program's state} is the contents of all memory locations
    referenced by \textbf{variables}.
  \item \textbf{Variable} is a pair consisting of a storage location and a
    \textbf{symbolic name} that can be used to retrieve information stored in
    that storage location.
  \end{itemize}
\end{frame}

\begin{frame}[fragile]
  \frametitle{Variable Scope}
  \begin{itemize}
  \item \textbf{Scope} refers to the textual region of the program, where
    references to some entity may occur.
  \item Common Lisp distinguishes the following:
    \begin{itemize}
    \item \textbf{lexical scope} - references to the entity may only occur
      in the part of the program that is lexically (textually) contained within
      the construct that establishes it,
    \item \textbf{indefinite scope} - references may occur anywhere in the
      program,
    \end{itemize}
  \end{itemize}
\end{frame}

\begin{frame}[fragile]
  \frametitle{Variable Extent}
  \begin{itemize}
  \item \textbf{Extent} refers to the interval of time during which the
    references to some entity may occur. 
  \item Common Lisp distinguishes the following:
    \begin{itemize}
    \item \textbf{dynamic extent} - references may only occur between the time
      control flow enters the construct that establishes the extent and exits
      it,
    \item \textbf{indefinite extent} - entity continues to exist while there is
      still a possibility of reference.
    \end{itemize}
  \end{itemize}
\end{frame}

\begin{frame}[fragile]
  \frametitle{Variables in Common Lisp 1/2}
  \begin{itemize}
  \item Associations between the symbolic names of the variables and their
    values are called \textbf{bindings}.
  \item Variables can be \textbf{bound}, meaning there is a binding for that
    variable at a current point in the program, and \textbf{free}, meaning there
    is no binding.
  \item Variable bindings are stored in objects called \textbf{environments},
    which can be nested:
\begin{minted}{Lisp}
                       ;; Environment 0:
  (let ((a 5)          ;; a = 5
        (b 10))        ;; b = 10
    ...
    ...                ;; Environment 1:
    (let ((c (+ a b))  ;; c = 15
          (d (* a b))) ;; d = 50
      ...
      (list a b c d))) ;; Lookup in Env. 1, then Env. 0
\end{minted}
  \end{itemize}
\end{frame}

\begin{frame}[fragile]
  \frametitle{Variables in Common Lisp 2/2}
  \begin{itemize}
  \item Variables can be set with the \mintinline{Lisp}{setf} form:
\begin{minted}{Lisp}
  CL-USER> (let ((a 0))
             (setf a 1)
             a)
  1
\end{minted}
  \item In the body of a function definition, variables representing the
    parameters are considered bound,
\begin{minted}{Lisp}
  (defun f (a b) 
    (* a b))     ;; a and b are bound to the abstract "arguments"

  (f 1 2)
  ;; The same as (almost, because in general variable
  ;; renaming is needed):
  ;; (let ((a 1)
  ;;       (b 2))
  ;;   (* a b))
\end{minted}
  \end{itemize}
\end{frame}


\begin{frame}[fragile]
  \frametitle{Lexically-scoped Variables 1/2}
  \begin{itemize}
  \item Simplest Common Lisp construct to create lexical bindings is
    a \mintinline{Lisp}{let}-form:
\begin{minted}{Lisp}
  CL-USER> (let ((a 5)
                 (b 6))
             (* a b))
  30
  CL-USER> (* a b)
  ;; Error: Attempt to take the value of the unbound variable `a'.
\end{minted}
  \item Here, variables \mintinline{Lisp}{a} and \mintinline{Lisp}{b} can only
    by accessed from within the body of the \mintinline{Lisp}{let}-form.
  \item Scope of variables \mintinline{Lisp}{a} and \mintinline{Lisp}{b} is the
    lexical (textual) region \mintinline{Lisp}{"(* a b)"}.
  \item Extent of variables \mintinline{Lisp}{a} and \mintinline{Lisp}{b} is the
    period while the form \mintinline{Lisp}{(* a b)} executes.
  \end{itemize}
\end{frame}

\begin{frame}[fragile]
  \frametitle{Lexically-scoped Variables 2/2}
  \begin{itemize}
  \item Lexical variables are invisible outside the construct that defines their
    scope:
\begin{minted}{Lisp}
  CL-USER> (define f ()
             (+ a b))

  CL-USER> (let ((a 5)
                 (b 6))
             (f))
  ;; Error: Attempt to take the value of the unbound variable `a'.
\end{minted}
  \end{itemize}
\end{frame}

\begin{frame}[fragile]
  \frametitle{Dynamically-scoped Variables}
  \begin{itemize}
  \item \textbf{``Dynamic scope''} is a hacky way of saying ``indefinite scope
    and dynamic extent''.
  \item Common Lisp allows to declare variables to be ``dynamically scoped'':
\begin{minted}{Lisp}
  (defvar *coeff* 10) ; "special" (dynamically scoped) variable 

  (defun f (x) (* x *coeff*))

  CL-USER> (f 5)
  50
  CL-USER> (let ((*coeff* 1000))
             (f 5))
  5000
  CL-USER> *coefficient*
  10
\end{minted}
  \item Note that \mintinline{Lisp}{let} can also create dynamic bindings.
  \end{itemize}
\end{frame}

\begin{frame}[fragile]
  \frametitle{Practical Example of Dynamically-scoped Variables}
  \begin{itemize}
  \item Dynamic variables allow to implicitly parameterize code without having
    to pass around lots of arguments:
\begin{minted}{Lisp}
  (defun print-numbers (n1 n2 n3)
    (print n1)
    (print n2)
    (print n3))

  (print-numbers 10 11 12)
  ;; 10
  ;; 11
  ;; 12

  (let ((*print-base* 16))
    (print-numbers 10 11 12))
  ;; a
  ;; b
  ;; c
\end{minted}
  \end{itemize}
\end{frame}

\begin{frame}[fragile]
  \frametitle{Control Flow 1/4: sequence of forms}
  \begin{itemize}
  \item Multiple Common Lisp forms can be executed in a sequence:
\begin{minted}{Lisp}
  (progn
     <form1>
     <form2>
     ...
     <formN>)
\end{minted}
  \item The value of the \mintinline{Lisp}{progn} form is the value of the last
    form:
\begin{minted}{Lisp}
  (progn
    (print "This will return 5.")
    5)
\end{minted}
  \item The order can be switched with \mintinline{Lisp}{prog1}:
\begin{minted}{Lisp}
  (prog1 5
    (print "This will return 5."))
\end{minted}
  \item A lot of Lisp forms that have a body, have an implicit
    \mintinline{Lisp}{progn} around the body.
  \end{itemize}
\end{frame}

\begin{frame}[fragile]
  \frametitle{Control Flow 2/4: conditionals}
  \begin{itemize}
  \item \mintinline{Lisp}{if}-expression has the following form:
\begin{minted}{Lisp}
  (if <condition>
      <true-branch-form>
      <false-branch-form>)
\end{minted}
    
  \item There is a single-branch conditional:
\begin{minted}{Lisp}
  (when <condition>
    <true-branch-body>
    ...)
  ;; (if <condition>
  ;;     (progn <true-branch-body>
  ;;            ...)
  ;;     nil
\end{minted}
    and a shortcut
\begin{minted}{Lisp}
  (unless <condition>   ;; same as (when (not <condition>) ...)
    <false-branch-body>
    ...)
\end{minted}
  \end{itemize}
\end{frame}

\begin{frame}[fragile]
  \frametitle{Control Flow 3/4: multiple values}
  \begin{itemize}
  \item Common Lisp allows to return multiple values as a result of evaluating
    a form:
\begin{minted}{Lisp}
  CL-USER> (values 1 2 3)
  1
  2
  3
  CL-USER> (multiple-value-bind (a b c) (values 1 2 3)
             (list 1 2 3))
  (1 2 3)
  CL-USER (let ((a (values 1 2 3)))
            a)
  1
\end{minted}
  \item Used when the rest of the values are less significant than the main one.
  \item Unlike Python's \mintinline{Python}{return a, b}, Common Lisp's
    \mintinline{Lisp}{values} form does not pack/unpack values into a transient
    data structure.
  \end{itemize}
\end{frame}

\begin{frame}[fragile]
  \frametitle{Control Flow 4/4: simple iteration}
  \begin{itemize}
  \item Common Lisp provides simple constructions to iterate over a list and
    repeat a body several times:
\begin{minted}{Lisp}
  CL-USER> (dolist (element '(1 2 3))
             (print element))
  1
  2
  3
  nil
  CL-USER> (dotimes (i 3)
             (print i))
  0
  1
  2
  nil
\end{minted}
  \item There are also more complex \mintinline{Lisp}{do} and
    \mintinline{Lisp}{loop} forms.
  \end{itemize}
\end{frame}

\begin{frame}[fragile]
  \frametitle{Lexical Closures 1/2}
  \begin{itemize}
  \item \textbf{Closure} is a pair consisting of a first-class function and an
    environment.
  \item Closure's \textbf{environment} maps the free variables in the function
    to the values they were bound to when the closure was created.
  \item Closure, unlike a regular function, can access the bindings in the
    environment where they were created.
\begin{minted}{Lisp}
  (defun make-counter ()
    (let ((c 0))
      (lambda () 
        (prog1 c
          (incf c))))) ;; same as (setf c (+ c 1))

  (let ((counter (make-counter)))
    (funcall counter)  ;; 0
    (funcall counter)) ;; 1
\end{minted}
  \end{itemize}
\end{frame}

\begin{frame}[fragile]
  \frametitle{Lexical Closures 2/2}
  \begin{itemize}
  \item Practical example of using lexical closures - \textbf{memoization}
\begin{minted}{Lisp}
  (let ((cache (make-hash-table)))

    (defun %fib (n)
      (if (= n 0)
          0
          (if (= n 1)
              1
              (+ (%fib (- n 1)) (%fib (- n 2))))))

    (defun fib (n)
      (multiple-value-bind (value exists?) (gethash n cache)
        (if exists?
            value
            (let ((value (%fib n)))
              (setf (gethash n cache) value)
              value)))))
\end{minted}
  \end{itemize}
\end{frame}

\begin{frame}[fragile]
  \frametitle{Lexical Closures 2/2}
  \begin{itemize}
  \item Practical example of using lexical closures - \textbf{memoization}
\begin{minted}{Lisp}
  (let ((cache (make-hash-table)))

    (defun %fib (n)
      (if (= n 0)
          0
          (if (= n 1)
              1
              (+ (fib (- n 1)) (fib (- n 2))))))

    (defun fib (n)
      (multiple-value-bind (value exists?) (gethash n cache)
        (if exists?
            value
            (let ((value (%fib n)))
              (setf (gethash n cache) value)
              value)))))
\end{minted}
  \end{itemize}
\end{frame}

% \begin{frame}[fragile]
%   \frametitle{Function Call Semantics}
%   \begin{itemize}
%   \item Function calls can be emulated with the \mintinline{Lisp}{let}-form:
%     assuming the following function definition
% \begin{minted}{Lisp}
%   (defun f (a b) (* a b))
% \end{minted}
%     function call
% \begin{minted}{Lisp}
%   (f 5 6)
% \end{minted}
%     is semantically equivalent to
% \begin{minted}{Lisp}
%   (let ((a 5)
%         (b 6))
%     (* a b)) 
% \end{minted}
%   \end{itemize}
% \end{frame}

\begin{frame}
  \frametitle{Useful Resources}
  \begin{itemize}
  \item Common Lisp HyperSpec - publicly available alternative to ANSI Common
    Lisp standard
    \begin{itemize}
    \item \scriptsize{http://www.lispworks.com/documentation/HyperSpec/Front/index.htm}
    \end{itemize}
  \end{itemize}
\end{frame}

\begin{frame}
  \frametitle{The End}
  \begin{center}
    Thank you!
  \end{center}
\end{frame}

\end{document}

%%% Local Variables:
%%% mode: latex
%%% TeX-master: t
%%% TeX-command-extra-options: "-shell-escape"
%%% End:
